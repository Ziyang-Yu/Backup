% !Mode:: "TeX:UTF-8"
% !TEX program  = xelatex
% !BIB program  = biber
\documentclass[AutoFakeBold,AutoFakeSlant,language=chinese,degree=bachelor]{sustechthesis}
% 1. AutoFakeBold 与 AutoFakeSlant 为伪粗与伪斜,如果本机上有相应粗体与斜体字体,请使用 xeCJK 宏包进行设置,例如:
%   \setCJKmainfont[
%     UprightFont = * Light,
%     BoldFont = * Bold,
%     ItalicFont = Kaiti SC,
%     BoldItalicFont = Kaiti SC Bold,
%   ]{Songti SC}
%
% 2. language=chinese 基于为 ctexart 文类提供的中文排版方案修改,如果使用英文进行论文创作,请使用 language=english 选项。
%
% 3. degree=bachelor 为 sustechthesis 文类提供的本科生毕业论文模板,其他可选项为 master 与 doctor,但是均未实现,如果您对此有兴趣,欢迎 PR。
%
% 4. sustechthesis.cls 文类主要参考自去年完成使命的 sustechthesis.tex,在这一年的时间,作者的 TeX 风格与常用宏包发生许多变化,因为之前的思想为仅提供必要的格式修改相关代码,所以转换为文类形式所进行的修改较少,而近期的风格与常用宏包均体现在以下的例子文件中。
%
% 5. 示例文件均放置于相应目录的 examples 文件夹下,构建自己论文时可暂时保留,用以检索接口与使用方法。
%
% 6. 英文目录需要居中可以使用:\renewcommand{\contentsname}{\centerline{Content}}
%
% 7. LaTeX 中公式编号括号样式及章节关联的方法:https://liam.page/2013/08/23/LaTeX-Formula-Number/

% !Mode:: "TeX:UTF-8"
% !TEX program  = xelatex

% 数学符号与环境
\usepackage{amsmath,amssymb}
  \newcommand{\dd}{\mathrm{d}}
  \newcommand{\RR}{\mathbb{R}}
% 参考文献
\usepackage[style=gb7714-2015]{biblatex}
  \addbibresource{ref.bib}
% 无意义文本
\usepackage{zhlipsum,lipsum}
% 列表环境设置
\usepackage{enumitem}
% 浮动题不越过 \section
\usepackage[section]{placeins}
% 超链接
\usepackage{hyperref}
% 图片,子图,浮动题设置
\usepackage{graphicx,subcaption,float}
% 抄录环境设置,更多有趣例子请命令行输入 `texdoc tcolorbox`
\usepackage{tcolorbox}
  \tcbuselibrary{xparse}
  \DeclareTotalTCBox{\verbbox}{ O{green} v !O{} }%
    {fontupper=\ttfamily,nobeforeafter,tcbox raise base,%
    arc=0pt,outer arc=0pt,top=0pt,bottom=0pt,left=0mm,%
    right=0mm,leftrule=0pt,rightrule=0pt,toprule=0.3mm,%
    bottomrule=0.3mm,boxsep=0.5mm,bottomrule=0.3mm,boxsep=0.5mm,%
    colback=#1!10!white,colframe=#1!50!black,#3}{#2}%
\tcbuselibrary{listings,breakable}
  \newtcbinputlisting{\Python}[2]{
    listing options={language=Python,numbers=left,numberstyle=\tiny,
      breaklines,commentstyle=\color{white!50!black}\textit},
    title=\texttt{#1},listing only,breakable,
    left=6mm,right=6mm,top=2mm,bottom=2mm,listing file={#2}}
% 三线表支持
\usepackage{booktabs}

% LaTeX logo
\usepackage{hologo}
 % 导言区
% !Mode:: "TeX:UTF-8"
% !TEX program  = xelatex
\设置信息{
    % 键 = {{中文值}, {英文值}},
    分类号 = {{}, {}},
    编号 = {{}, {}},
    UDC = {{}, {}},
    密级 = {{}, {}},
    % 仅题目(不含副标题)、系别、专业,支持手动 \\ 换行,不支持自动换行。
    题目 = {{改善滞后性的图神经网络训练框架}, {Staleness-Alleviated GNN Training}},
    % 如无需副标题,删除值内容即可,不可删除键定义。
    副标题 = {{}, {}},
    姓名 = {{喻子洋}, {Ziyang Yu}},
    学号 = {{11910419}, {11910419}},
    系别 = {{数学系}, {Department of Mathematics}},
    专业 = {{数学与应用数学}, {Mathematics and Applied Mathematics}},
    指导老师 = {{张振}, {Zhen Zhang}},
    时间 = {{2023年4月15日}, {April 15, 2023}},
    职称 = {{副教授}, {Associate Professor}},
}
 % 论文信息
\begin{document}

\中文标题页
%\英文标题页

\中文诚信承诺书
%\英文诚信承诺书

\前序格式化
\摘要标题
% !Mode:: "TeX:UTF-8"
% !TEX program  = xelatex
\begin{中文摘要}{\LaTeX ;接口}
  尽管最近GNN取得了成功,但在具有数百万节点和数十亿条边的大型图上训练GNN仍然具有挑战性,这在许多现实世界的应用中是普遍存在的。在分布式GNN训练中,"基于分区 "的方法享有较低的通信成本,但由于掉落的边而遭受信息损失,而 "基于传播 "的方法避免了信息损失,但由于邻居爆炸而遭受令人望而却步的通信开销。这两者之间的一个自然的 "中间点 "是通过缓存历史信息(例如,节点嵌入),这可以实现恒定的通信成本,同时保留不同分区之间的相互依赖。然而,这样的好处是以涉及过时的训练信息为代价的,从而导致呆板性、不精确性和收敛问题。为了克服这些挑战,本文提出了SAT(滞后性缓解训练),这是一个新颖的、可扩展的分布式GNN训练框架,可以自适应地减少嵌入的滞后性。具体来说,我们提出将嵌入预测建模为学习由历史嵌入引起的时间图,并建立一个子图不变的弱监督辅助模块来预测未来的嵌入,该模块以数据驱动的方式减少呆滞性,并具有良好的可扩展性。预测的嵌入作为一个桥梁,以在线的方式交替训练分布式GNN和辅助模块。最后,我们提供了广泛的理论和经验分析
  来证明所提出的方法可以在性能和收敛速度上大大改善现有的框架,而只需最小的额外费用。
  
\end{中文摘要}

\begin{英文摘要}{LaTeX, Interface}
  \lipsum[1]
\end{英文摘要}
 % 论文摘要

\目录\clearpage % 目录及换页

\正文格式化
\begin{section}{简介}
    
图神经网络在分析非欧几里得图数据方面取得了令人印象深刻的成功,并在各种应用中取得了可喜的成果,包括社交网络、推荐系统和知识图等~\cite{dai2016discriminative,ying2018graph,lei2019gcn}。尽管GNN有很大的前景,但在应用于现实世界中常见的大型图时,GNN遇到了巨大的挑战--大型图的节点数量可以达到数百万甚至数十亿。
在大型图上训练GNN,由于缺乏内在的并行性,共同面临着挑战。
在大型图上训练GNN,由于反向传播优化中缺乏固有的并行性,以及图节点之间严重的相互依赖性,共同构成了挑战。
为了应对这种独特的挑战,分布式GNN训练是一个很有前景的开放领域,近年来吸引了越来越多的关注~\cite{dorylus_osdi21,ramezani2021learn,wan2022pipegcn,chai2022distributed},并已成为在大型图上快速准确训练的标准。



现有的分布式GNN训练方法可以分为两类,即"\emph{基于分割}"和"\emph{基于传播}",根据它们如何解决\emph{计算/通信成本}和\emph{信息损失}之间的权衡。"基于分割"方法~\cite{angerd2020distributed,jia2020improving,ramezani2021learn}通过丢弃跨子图的边,将图分割成不同的子图,因此大图上的GNN训练被分解成多个较小的训练任务,每个训练任务在一个孤立的子图中并行,减少子图间的通信。
然而,这将导致严重的信息损失,因为对子图之间节点的依赖关系一无所知,并导致性能下降。
为了减轻信息损失,"基于传播 "的方法~\cite{ma2019neugraph,zhu2019aligraph,zheng2020distdgl,tripathy2020reducing}不忽略不同子图之间的边,子图之间的邻居通信,以满足GNN的邻居聚合。然而,随着GNN的深入,参与邻居聚合的邻居数量呈指数级增长(即, \emph{邻居爆炸}~\cite{hamilton2017inductive}),因此不可避免地遭受巨大的通信开销和困扰的效率。
打破基于分区和传播的方法的悖论的一个自然方法是通过增加第三个维度,形成一个 \emph{抵消三角}。具体来说,通过利用离线存储器来存储子图外邻居的 \emph{历史} 嵌入,并在下一个阶段需要时将其拉到GPU上,可以实现相对于节点数量的恒定通信成本,同时保留子图之间的相互依赖关系。这种技术~\cite{chen2018stochastic,fey2021gnnautoscale,wan2022pipegcn,chai2022distributed}已被广泛用于分布式GNN训练,并取得了最先进的性能。
\end{section}
\begin{section}{背景}
    \textbf{图形神经网络}。GNN的目的是在一个具有节点表征
    $\textit{X}\in\textit{R}^{\vert\textit{V}\vert\times d}$
    的图上学习信号/特征的函数,其中$d$表示节点特征维度。对于典型的半监督节点分类任务~\cite{kipf2016semi},其中每个节点都与一个标签相关、 一个层的GNN 参数化为被训练来学习节点表征,这样可以被准确预测。GNN的训练过程实际上可以描述为基于\textit{消息传递机制}~\cite{gilmer2017neural}的节点表示学习。从分析上看,给定一个图和一个节点,GNN的第三层被定义为
    \begin{equation}
        \label{eq:gnn}
        \vspace{-1mm}
        \begin{split}
            \mathbf{h}_{v}^{(\ell+1)} &= \boldsymbol{f}_{\boldsymbol{\theta}}^{(\ell+1)} \bigg(\mathbf{h}_{v}^{(\ell)}, \big\{\!\!\big\{ \mathbf{h}^{(\ell)}_u  \big\}\!\!\big\}_{u \in \mathcal{N}(v)} \bigg) \\
            &= \Psi^{(\ell+1)}_{\boldsymbol{\theta}}\bigg( \mathbf{h}_{v}^{(\ell)},  \Phi^{(\ell+1)}_{\boldsymbol{\theta}}\Big(\big\{\!\!\big\{\mathbf{h}^{(\ell)}_u  \big\}\!\!\big\}_{u \in \mathcal{N}(v)}\Big) \bigg),
        \end{split}
        \vspace{-2mm}
        \end{equation}
    其中,$\textit{h}_{v}^{(\textit{l})}$表示节点$v$在$\ell$第三层的表示, $\textbf{h}^{(0)}_{v}$ 被初始化为 $\textbf{x}_{v}$($\textbf{X}$中第$v$行), $mathcal{N}(v)$表示节点$v$的邻居集合。
    GNN的每一层,即$\boldsymbol{f}_{\boldsymbol{\theta}}^{(\ell)}$,可以进一步分解为两个部分:
    1) 聚合函数$\Phi^{(\ell)}_{\boldsymbol{\theta}}$,它将节点$v$的邻居的节点表示作为输入,输出聚合的邻居表示。
    2) 更新函数$\Psi^{(\ell)}_{\boldsymbol{\theta}}$,它结合$v$的表示和聚集的邻域表示,为下一层更新节点$v$的表示。
    $\Phi^{(\ell)}_{\boldsymbol{\theta}}$和$\Psi^{(\ell)}_{\boldsymbol{\theta}}$都可以选择在不同类型的GNN中使用各种函数。
    为了在一台机器上训练GNN,我们可以在训练数据的整个图上使经验损失$\mathcal{L}(\boldsymbol{\theta})$最小化,即$\mathcal{L}(\boldsymbol{\theta}) = \vert\mathcal{V}\vert^{-1}\sum\nolimits_{v\in\mathcal{V}}。Loss\big(\mathbf{h}_{v}^{(L)},\mathbf{y}_{v}\big)$,
    其中$Loss(\cdot,\cdot)$表示损失函数(如交叉熵损失),$\mathbf{h}_{v}^{(L)}$表示来自GNN最后一层的节点$v$的表示。
\end{section}


\begin{section}{问题描述}
    \textbf{分布式GNN训练下的时态图预测.} 我们遵循Eq.~\ref{eq:mini-batch}和Eq.~\ref{eq:distributed loss}中定义的GNN的分区并行分布式训练。考虑到任何分区$\mathcal{G}_m$,我们将Eq.~\ref{eq:mini-batch}以矩阵形式重新表述为
    
    \begin{equation}
        \mathbf{H}_{in}^{(\ell+1,m)} = \boldsymbol{f}_{\boldsymbol{\theta}}^{(\ell+1)}\Big(\mathbf{H}_{in}^{(\ell,m)},\mathbf{H}_{out}^{(\ell,m)}\Big),
    \label{eq:matrix form forward}
    \end{equation}

    其中$\mathbf{H}_{in}^{(\ell,m)}$和$\mathbf{H}_{out}^{(\ell,m)}$分别表示分区$\mathcal{G}_m$上第$\ell$层的子图内和子图外节点嵌入矩阵。如前所述,直接在每个子图之间交换$\mathbf{H}_{out}^{(\ell,m)}$ 将导致指数级的通信成本。现有的方法通过存储在离线存储器中的历史值对$\mathbf{H}_{out}^{(\ell,m)}$进行近似,即 $\mathbf{H}_{out}^{(\ell,m)} \approx \mathbf{\Tilde{H}}_{out}^{(\ell,m)}$, 这导致了$\delta\mathbf{H}^{(\ell,m)} = \mathbf{H}_{out}^{(\ell,m)} - \mathbf{\Tilde{H}}_{out}^{(\ell,m)}$。相反,我们考虑以数据驱动的方式预测$\mathbf{H}_{out}^{(\ell,m)}$ ,使预测的嵌入$\mathbf{\hat{H}}_{out}^{(\ell,m)}$具有较小的滞后性。

    在这项工作中,我们创新性地将分布式GNN训练下的嵌入预测任务表述为对时态图演化模式的建模。

    具体来说,考虑$\mathcal{\Bar{G}}_{m}= (\mathcal{\Bar{V}}_{m},\mathcal{\Bar{E}}_{m})$的子图,该子图由$\mathcal{G}^{(t)}_{m}$及其1-\emph{hop}邻居诱导。
由于$\mathbf{H}_{in}^{(m)}$和$\mathbf{H}_{out}^{(m)}$ 在训练过程中不断变化(即、 在历时中),图的序列$\mathcal{\Bar{G}}^{(t)}_{m}= (\mathcal{\Bar{V}}_{m},\mathcal{\Bar{E}}_{m},\mathbf{H}_{in}^{(t,m)},\mathbf{H}_{out}^{(t,m)})$,$t=1,2,\cdots,T$成为一个具有固定拓扑结构的\emph{时空图}~\cite{wu2022graph},其中$T$表示历时的总数。

我们的目标是建立一个模型,积极主动地捕捉不断变化的模式,并以在线方式预测嵌入。详细来说,给定时间图$\{\mathcal{\Bar{G}}^{(s)}: t-\tau\leq s\leq t\}$到历时$t$,其中$\tau$表示滑动窗口长度,我们学习一个映射函数$M_{\boldsymbol{\omega}}$,参数为$\boldsymbol{\omega}$,这样它可以预测下一个历时的子图外嵌入$\mathbf{\hat{H}}_{out}^{(t+1,m)}$。对$M_{\boldsymbol{\omega}}$的训练和推理在每个历时$t=1,2,\cdots,T-1$进行一次。

尽管有此必要,但由于现有的几个挑战,如何处理上述问题是一个开放的研究领域: 1). 难以设计出一种高效的、可扩展的减少呆滞性的算法。2). 嵌入的未知和复杂演变。3).由于嵌入的未知动态性,在理论上很难获得对具有预测嵌入的分布式GNN框架的性能和收敛性的保证。

\textbf{训练目标}。为了有效而廉价地获得辅助模型的监督,我们提出了一种弱监督的误差回归损失来训练辅助模型,它带来的额外计算成本几乎为零。具体来说,对于任何子图外的节点$u \in \mathcal{N}(v) \setminus \mathcal{G}_{m}$,一定存在另一个子图$\mathcal{G}_{n}$,使得$u \in \mathcal{G}_{n}$(每个节点必须属于某些图的分区)。在子图$\mathcal{G}_{n}$上用$u$替换节点$v$,在Eq.~\ref{eq:synthetic forward}中可以得到$\mathbf{h}_{u}^{(\ell)}$,其中的计算在GNN模型的前向传播中是/emph{hidden}。用$\mathbf{h}_{u}^{(\ell)}$ 减去$\mathbf{\Tilde{h}}^{(\ell)}_u$ 的结果是地 真相嵌入误差 $\Delta\mathbf{h}^{(\ell)}_u = \mathbf{h}_{u}^{(\ell)} - \mathbf{\Tilde{h}}^{(\ell)}_u$、 而辅助模型$M_{\ell}(\cdot)$的训练损失可以定义为$M_{\ell}(\cdot)$, 
\begin{equation}
    \min\;\; \sum\nolimits_{u\in\mathcal{G}} \big\Vert \Delta\mathbf{\hat{h}}^{(\ell)}_u - \Delta\mathbf{h}^{(\ell)}_u \big\Vert_{2}^{2},
\label{eq:aux model loss}
\end{equation}
其中最小化是在$M_{\ell}$的参数上。我们的上述目标函数可以通过从$\mathcal{G}$中取样的小批次轻松地扩展到随机训练,这就允许了\emph{可控}的训练样本量,这取决于辅助模型的效率和性能之间的权衡。

\textbf{备注} \textbf{1).}。如公式所示~\ref{eq:aux def 1},辅助模型借用了skip-connection~\cite{he2016identity}的思想,辅助模型的预测可以解释为一个错误\emph{修正}项,它纠正了前一时代嵌入的呆滞性。\textbf{2).}。我们在Eq.~\ref{eq:synthetic forward}中的表述是相当普遍的,而且是与模型无关的。现有的具有历史嵌入的GNN训练框架~\cite{chen2018stochastic,fey2021gnnautoscale,wan2022pipegcn}可以享受我们的预测嵌入来进一步提升他们的性能。






\end{section}
\begin{section}{解决方法}
    在这一节中,我们介绍了我们提出的呆滞性-减弱分布式GNN训练框架,以解决上述挑战。对于第一个挑战,我们建立了一个子图不变的弱监督辅助模块,用于未来的嵌入预测,该模块以数据驱动的方式减少呆滞性,并具有良好的可扩展性。对于第二个挑战,我们建议将GNN与递归结构结合起来,这样前者可以捕获节点连接中的信息,后者可以捕获时间演变的信息。同时,我们的在线设计允许辅助模块和分布式GNN交替训练,并使之相互受益。我们通过引入下采样预训练和自适应微调频率,进一步提高辅助模块的性能和效率。最后,为了应对最后的挑战,我们探讨了所引入的辅助模块在分布式环境下如何影响模型性能和收敛的理论保证。


    \begin{subsection}{框架} 
        在本地机器上训练的GNN的每个副本将利用所有可用的图信息,即在前向和后向传播中都放弃了边缘。从分析上看,计算每个本地梯度$\nabla\mathcal{L}_{m}^{\text{Local}}$,如公式~\ref{eq:distributed loss}中定义的,将涉及子图外的邻居信息。
        对于子图外的邻居节点,我们考虑使用由辅助模型(表示为 $\mathbf{\hat{h}}^{(\ell)}_{v}$)预测的\emph{predicted embedding}代替真实嵌入。形式上,给定$v\in\mathcal{G}_{m}(\mathcal{V}_{m},\mathcal{E}_{m})$ 的节点来自第$m$个子图,通过修改Eq.~\ref{eq:mini-batch}实现前向传播。
        \begin{equation}
            \begin{split}
            \mathbf{h}_{v}^{(\ell+1)} = \boldsymbol{f}^{(\ell+1)}_{\boldsymbol{\theta}}\Big( \mathbf{h}_{v}^{(\ell)}, &\big\{\!\!\big\{\mathbf{h}^{(\ell)}_u \big\}\!\!\big\}_{u \in  \mathcal{N}(v)   \cap \mathcal{G}_{m}} \\ &\cup \underbrace{\big\{\!\!\big\{\mathbf{\hat{h}}^{(\ell)}_u  \big\}\!\!\big\}_{u \in \mathcal{N}(v) \setminus \mathcal{G}_{m}}}_{\text{predicted embedding}} \Big),
        \end{split}
        \label{eq:synthetic forward}
        \end{equation}
    \end{subsection}
    其中 $\forall$ $u \in \mathcal{N}(v) \setminus \mathcal{G}_{m}$ 满足

    \begin{equation}
        \begin{split}
            \mathbf{\hat{h}}^{(\ell)}_u = \mathbf{\Tilde{h}}^{(\ell)}_u + \Delta\mathbf{\hat{h}}^{(\ell)}_u,\quad\Delta\mathbf{\hat{h}}^{(\ell)}_u = M_{\ell}\big(\mathbf{\Tilde{h}}^{(\ell)}_u, \xi\big).
        \end{split}
    \label{eq:aux def 1}
    \end{equation}

    这里,$\mathbf{\Tilde{h}}^{(\ell)}_u$表示前一个历时的嵌入,$M_{\ell}(\cdot)$表示$\ell$第三层的辅助模型。$\xi$表示任何其他历史信息作为$M_{\ell}(\cdot)$的额外输入。如公式~\ref{eq:aux def 1}所示,预测的嵌入等同于前一纪元的嵌入加上一个误差项,这个误差项是由辅助模型仅根据历史信息预测的。

    请注意,在Eq~\ref{eq:aux def 1}中,我们的辅助模型是在不同的子图中共享的($M_{\ell}(\cdot)$不含$m$)。原因有3个方面: \textbf{1).}。从理论上讲,在分割之前,所有的子图都来自同一个全局输入图,因此共享相同的嵌入分布。我们通过在不同的子图/分区中共享我们的辅助模块来编码这种归纳偏见,这使得辅助模块能够捕捉重要的或重复的模式,而不考虑子图的差异。 \textbf{2).}。实际上,由于过度平滑的问题,GNN模型的层次相对较浅~\cite{chen2020measuring}。另一方面,现代图可以轻易地超过数百万个节点。为了处理这样巨大的图,人们必须将其划分为大量的子图,以便将每个子图装入一个GPU。因此,在分布式GNN问题中,我们通常有$M\gg L$,所以一个更有效的内存方式是在不同的子图上共享辅助模型。\textbf{3).}。更重要的是,这种共享策略还可以增加辅助模型的训练样本量,从而提高辅助模型的\emph{generalizability}。如公式~\ref{eq:aux model loss}所示,$M_{\ell}$享有整个输入图$\mathcal{G}$的训练样本量(即$\vert\mathcal{V}(\mathcal{G})\vert$),而针对分区的辅助模型,即 $M_{\ell,m}$ 只能有最多$\big\vert\mathcal{V}\big(\mathcal{N}(\mathcal{G}_{m}) \setminus \mathcal{G}_{m}\big)\big\vert$样本来学习,这是子图$\mathcal{G}_m$的1-emph{hop}邻居数量。


\end{section}\clearpage
%% !Mode:: "TeX:UTF-8"
% !TEX program  = xelatex
\begin{中文摘要}{\LaTeX ;接口}
  尽管最近GNN取得了成功,但在具有数百万节点和数十亿条边的大型图上训练GNN仍然具有挑战性,这在许多现实世界的应用中是普遍存在的。在分布式GNN训练中,"基于分区 "的方法享有较低的通信成本,但由于掉落的边而遭受信息损失,而 "基于传播 "的方法避免了信息损失,但由于邻居爆炸而遭受令人望而却步的通信开销。这两者之间的一个自然的 "中间点 "是通过缓存历史信息(例如,节点嵌入),这可以实现恒定的通信成本,同时保留不同分区之间的相互依赖。然而,这样的好处是以涉及过时的训练信息为代价的,从而导致呆板性、不精确性和收敛问题。为了克服这些挑战,本文提出了SAT(滞后性缓解训练),这是一个新颖的、可扩展的分布式GNN训练框架,可以自适应地减少嵌入的滞后性。具体来说,我们提出将嵌入预测建模为学习由历史嵌入引起的时间图,并建立一个子图不变的弱监督辅助模块来预测未来的嵌入,该模块以数据驱动的方式减少呆滞性,并具有良好的可扩展性。预测的嵌入作为一个桥梁,以在线的方式交替训练分布式GNN和辅助模块。最后,我们提供了广泛的理论和经验分析
  来证明所提出的方法可以在性能和收敛速度上大大改善现有的框架,而只需最小的额外费用。
  
\end{中文摘要}

\begin{英文摘要}{LaTeX, Interface}
  \lipsum[1]
\end{英文摘要}

%% !Mode:: "TeX:UTF-8"
% !TEX program  = xelatex
\section{免责声明}
\begin{enumerate}[label={\alph*)}]
    \item 本模板的发布遵守 \LaTeX\ Project Public License,使用前请认真阅读协议内容。
    \item 南方科技大学教学工作部只提供毕业论文写作指南,不提供官方模板,也不会授权第三方模板为官方模板,所以此模板仅为写作指南的参考实现,不保证格式审查老师不提意见. 任何由于使用本模板而引起的论文格式审查问题均与本模板作者无关。
    \item 任何个人或组织以本模板为基础进行修改,扩展而生成的新的专用模板,请严格遵守 \LaTeX\ Project Public License 协议。由于违犯协议而引起的任何纠纷争端均与本模板作者无关。
\end{enumerate}

%% !Mode:: "TeX:UTF-8"
% !TEX program  = xelatex
\section{文类接口}
文类的接口的命名均为汉字,意思为字面意思,如有疑问,欢迎在 GitHub 提出 \href{https://github.com/Iydon/sustechthesis/issues}{Issues}。

\subsection{汉化字号接口}
本接口主要使用 \texttt{ctex} 宏包。

\verbbox{\初号},\verbbox{\小初},\verbbox{\一号},\verbbox{\小一},\verbbox{\二号},\verbbox{\小二},\verbbox{\三号},\verbbox{\小三},\verbbox{\四号},\verbbox{\小四},\verbbox{\五号},\verbbox{\小五},\verbbox{\六号},\verbbox{\小六},\verbbox{\七号},\verbbox{\八号}。


\subsection{汉化字体接口}
可能本机上部分字体不存在,导致部分字体无法使用。

\verbbox{\宋体},\verbbox{\黑体},\verbbox{\仿宋},\verbbox{\楷书},\verbbox{\隶书},\verbbox{\幼圆},\verbbox{\雅黑},\verbbox{\苹方}。


\subsection{字体效果接口}

建议在正文时使用 \verb|\textbf{}|,\verb|\textit{}| 调用\textbf{粗体}与\textit{斜体}。

It is recommended to use \verb|\textbf{}|,\verb|\textit{}| to call \textbf{Bold} and \textit{ItalicFont}.

\verbbox{\粗体},\verbbox{\斜体}。


\subsection{格式相关接口}
\subsubsection{命令}
例子请参考前文,在写论文初期,可以注释掉标题页等不必要信息,以加快编译速度。

\verbbox{\设置信息},\verbbox{\目录},\verbbox{\下划线},\verbbox{\中文标题页},\verbbox{\英文标题页},\verbbox{\中文诚信承诺书},\verbbox{\英文诚信承诺书},\verbbox{\摘要标题},\verbbox{\参考文献},\verbbox{\附录},\verbbox{\致谢}。

\subsubsection{环境}
摘要环境均需一个参数,为关键词:\verb|\begin{}{}...\end{}|。

\verbbox{中文摘要},\verbbox{英文摘要}。

%% !Mode:: "TeX:UTF-8"
% !TEX program  = xelatex

\section{一些样例}

\subsection{表格}

表格与图片可以直接通过\verbbox{\ref{<key>}}来引用,例如表\ref{table2}、图\ref{F:test-a}、图\ref{F:test-b-sub-b}。

\begin{table}[htb]
% h-here,t-top,b-bottom,优先级依次下降
    \begin{center}
    % 居中
        \caption{表格的标题应该放在上方}\label{table}
        \begin{tabular}{lc} % 三线表不能有竖线,l-left,c-center,r-right
            \toprule
            %三线表-top 线
            Example & Result \\
            \midrule
            %三线表-middle 线
            Example1          & 0.25 \\
            Example2          & 0.36 \\
            \bottomrule
            %三线表-底线
        \end{tabular}
    \end{center}
\end{table}

\begin{table}[htb]
    \centering
    \caption{表格的标题}
    \label{table2}
    \begin{threeparttable}
        \setlength{\tabcolsep}{0.6cm}{ % 调节表格长度
                \begin{tabular}{lc} % 三线表不能有竖线,l-left,c-center,r-right
                    \toprule
                    %三线表-top 线
                    Example & Result \\
                    \midrule
                    %三线表-middle 线
                    Example1          & 0.25 \\
                    Example2          & 0.36 \\
                    \bottomrule
                    %三线表-底线
                \end{tabular}
        }
        \begin{tablenotes}
            \item[] 数据来源:% 增加表格数据来源注释
        \end{tablenotes}
    \end{threeparttable}
\end{table}

\subsection{参考文献}

参考文献一般使用\verbbox{\cite{<key>}}命令,效果如是\cite{Nicholas1998Handbook},引用作者使用\verbbox{\citeauthor{<key>}},效果如是“\citeauthor{goossens1994latex}”。

%\begin{figure}[htb]
    \centering
    \includegraphics[width=.5\textwidth]{example-image-a}
    \caption{自带测试图片---Test image}\label{F:test-a}
    % 图片的标题应该在下方
\end{figure}

\begin{figure}[htb]
    \centering
    \begin{subfigure}[t]{.45\linewidth}
        \centering
        \includegraphics[width=1\textwidth]{example-image-a}
        \caption{子图-自带测试图片---Test image}\label{F:test-b-sub-a}
    \end{subfigure}
    \begin{subfigure}[t]{.45\linewidth}
        \centering
        \includegraphics[width=1\textwidth]{example-image-a}
        \caption{子图-自带测试图片---Test image}\label{F:test-b-sub-b}
    \end{subfigure}
    \caption{自带测试图片---Test image}\label{F:test-b}
\end{figure}
%% !Mode:: "TeX:UTF-8"
% !TEX program  = xelatex
\section{\LaTeX\ 入门}
请参考 \href{https://tex.readthedocs.io/zh_CN/latest/}{在线文档},包括学习资源及学习路径。欢迎在 GitHub 上提出 \href{https://github.com/Iydon/tex/issues}{Issues}。
\clearpage

\参考文献
  \printbibliography[heading=none]\clearpage
\附录
  % !Mode:: "TeX:UTF-8"
% !TEX program  = xelatex
在这一节中,我们描述了详细的实验设置、额外的实验结果和完整的证明。
我们重新使用了从GNNAutoscale~\cite{fey2021gnnautoscale}中采用的部分代码;本文的代码可在: \url{https://github.com/Ziyang-Yu/SAT}。
请注意,为了提高可读性,对代码进行了重新组织。 

\section*{实验设置的细节}


所有的实验都是在AWS上的EC2 {texttt{g4dn.metal}}虚拟机(VM)实例上进行的,该实例拥有8美元的NVIDIA T4 GPU,96美元的vCPUs,384美元~GB的主内存。其他重要信息,包括操作系统版本、Linux内核版本和CUDA版本,在表~ref{tab:版本}中进行了总结。为了公平比较,我们对所有10个框架,PipeGCN、PipeGCN$^{+}$、GNNAutoscale、GNNAutoscale$^{+}$、DIGEST、DIGEST$^{+}$、VRGCN、VRGCN$^{+}$、DIGEST-A和DIGEST-A$^{+}$使用相同的优化器(亚当)、学习率和图划分算法。
对于PipeGCN、GNNAutoscale、DIGEST、VRGCN、DIGEST-A独有的参数,如每个节点从每层采样的邻居数和层数,我们为PipeGCN$^{+}$、GNNAutoscale$^{+}$、DIGEST$^{+}$、VRGCN$^{+}$和DIGEST-A$^{+}$选择相同值。
%/yuec{不是DGL??}。
这十个框架中的每一个都有一套专门针对该框架的参数;对于这些专属参数,我们对其进行调整,以达到最佳性能。请参考texttt{small\_benchmark/conf}下的配置文件,了解所有模型和数据集的详细配置设置。

我们使用七个数据集: Cora~\cite{sen2008collective}, Citeseer~\cite{sen2008collective}, Pubmed~\cite{sen2008collective}, OGB-Arxiv~\cite{hu2020open}, Flickr~\cite{zeng2019graphsaint}, Reddit~\cite{zeng2019graphsaint}, and OGB-Products~\cite{hu2020open}进行评估。这些数据集的详细信息在表~\ref{tab:dataset}中进行了总结。


\begin{table}[h]
    \centering
    \caption{我们测试平台的环境设置概要}
    \label{tab:versions}
    \resizebox{\textwidth}{!}{%
    \begin{tabular}{ccccccc}
    \hline
    OS    & Linux kernel & CUDA & Driver    & PyTorch & PyTorch Geometric & PyTorch Sparse \\ \hline
    Ubuntu~20.04 & 5.15.0  & 11.6 & 510.73.08 & 1.12.1  & 2.2.0          & 0.6.16     \\ \hline
    \end{tabular}%
    }
\end{table}




\begin{table}[h]
    \large
    \centering
    \caption{数据集概要}
    \label{tab:dataset}
    % \resizebox{\textwidth}{!}{%
    \begin{tabular}{llllll}
    \hline
    Dataset      & \#~Nodes   & \#~Edges     & \#~Features & \#~Classes \\ \hline
    Cora         & 2,708      & 10,556        & 1,433      & 7            \\
    Citeseer     & 3,327	     & 9,104	   & 3,703	    & 6            \\
    Pubmed       & 19,717	 & 88,648	   & 500	    & 3            \\
    Flickr       & 89,250    & 899,756     & 500        & 7            \\
    Reddit       & 232,965   & 23,213,838  & 602        & 41           \\
    OGB-Arixv    & 169,343   & 2,315,598   & 128        & 40           \\
    OGB-Products & 2,449,029 & 123,718,280 & 100        & 47           \\ \hline
    \end{tabular}%
    \end{table}


\clearpage
\致谢
  % !Mode:: "TeX:UTF-8"
% !TEX program  = xelatex
\sustechthesis\ 目前版本为 \version, \LaTeX\ 毕业论文模板项目从提出到现在已有两年了。感谢为本项目贡献代码的开发人员们:
\begin{itemize}
    \item 梁钰栋(南方科技大学,本科 17 级);
    \item 张志炅(南方科技大学,本科 17 级)。
\end{itemize}
以及使用本项目,并提出诸多宝贵的修改意见的使用人员们:
\begin{itemize}
    \item 李未晏(南方科技大学,本科 15 级);
    \item 张尔聪(南方科技大学,本科 15 级)。
\end{itemize}

此外,目前的维护者并非计算机系,可能存在对协议等的错误使用,如果你在本模板中发现任何问题,请在 GitHub 中提出 \href{https://github.com/Iydon/sustechthesis/issues}{Issues},同时也非常欢迎对代码的贡献!

\end{document}
