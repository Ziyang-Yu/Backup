% !Mode:: "TeX:UTF-8"
% !TEX program  = xelatex
\begin{中文摘要}{在线学习 ; 图神经网络 ; 分布式训练}
    尽管最近GNN取得了成功,但在具有数百万节点和数十亿条边的大型图上训练GNN仍然具有挑战性,这在许多现实世界的应用中是普遍存在的。在分布式GNN训练中,"基于分区 "的方法享有较低的通信成本,但由于掉落的边而遭受信息损失,而 "基于传播 "的方法避免了信息损失,但由于邻居爆炸而遭受令人望而却步的通信开销。这两者之间的一个自然的 "中间点 "是通过缓存历史信息(例如,节点嵌入),这可以实现恒定的通信成本,同时保留不同分区之间的相互依赖。然而,这样的好处是以涉及过时的训练信息为代价的,从而导致呆板性、不精确性和收敛问题。为了克服这些挑战,本文提出了SAT(滞后性缓解训练),这是一个新颖的、可扩展的分布式GNN训练框架,可以自适应地减少嵌入的滞后性。具体来说,我们提出将嵌入预测建模为学习由历史嵌入引起的时间图,并建立一个子图不变的弱监督辅助模块来预测未来的嵌入,该模块以数据驱动的方式减少呆滞性,并具有良好的可扩展性。预测的嵌入作为一个桥梁,以在线的方式交替训练分布式GNN和辅助模块。最后,我们提供了广泛的理论和经验分析
    来证明所提出的方法可以在性能和收敛速度上大大改善现有的框架,而只需最小的额外费用。
    
  \end{中文摘要}
  
  \begin{英文摘要}{LaTeX, Interface}
    \lipsum[1]
  \end{英文摘要}
  